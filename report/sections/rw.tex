%!TEX root = ../main.tex
\section{Random walks and preferential attachment}\label{section:random-walk}

\subsection{Theoretical degree distribution}
One of the weaknesses of the BA model and its generalizations is that this implicitly requires a knowledge of the total degree and a calculation across existing vertices on the graph. This requirements then destroys the potential for this model to exhibit emergent properties based on local behaviour. In real-world networks, such as social networks or webpages, the new 'vertices' that join rarely have a global knowledge of the other network vertices. The attachment by performing a random walk is a solution proposed by \citet{Saramaki2004}. In this model, a vertex is chosen at random from existing vertices and then executes a random walk of length $L$ from that vertex. The new vertex then attaches to the destination vertex. 

Preferential attachment then follows from the fact that the random walker is more likely to end up at a more highly connected vertex. This models real-world models such as interconnected webpages better, since we are likely to click on links to webpages from webpages we are already visiting. 

This model was thought to be able to reproduce the BA degree distribution even for $L=1$ \citep{Saramaki2004,J.P.Saramaki2004}. While this is the case for large $L$, \citet{Cannings2013} later showed that the $L=1$ degree sequence converges to a degenerate limiting solution in which almost every vertex has degree 1, instead of a power law distribution, and demonstrated that this model is fundamentally different from the BA model, unless we allow an indefinite length for the random walk. For $L = 0$, this reduces to the random attachment model. 

For L=1: likelihood ratio tests, clauset