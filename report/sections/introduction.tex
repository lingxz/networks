%!TEX root = ../main.tex
\section{Introduction}\label{section:introduction}

The Barab\'asi-Albert (BA) model is a model for generating scale-free networks with a preferential attachment mechanism \citep{Barabasi1999}. It has been extensively studied in many fields for its resemblance to real-world networks such as the internet and citation networks. The fat tail is a central characteristic in the model. Every new vertex attaches itself to existing vertices with a probability proportional to its degree, so that vertices with a higher degree are more likely to further increase its degree count, leading to a scale free degree distribution. 

There are two key parts to this model: growth and preferential attachment. The random attachment model is a limiting case of the BA model where it retains growth but does not include preferential attachment. Here we show that the resulting degree distribution in this limit is no longer scale-free but geometric. 

Recent works \citep{Saramaki2004,Cannings2013,J.P.Saramaki2004} have also investigated if preferential attachment can be reproduced without a global knowledge of the entire network, by performing a random walk on th graph. In this project we briefly look at the of such a model and whether it reproduces the power law for preferential attachment. 